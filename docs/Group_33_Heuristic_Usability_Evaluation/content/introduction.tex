% !TEX root =  ../report.tex
\section{Introduction}

Heuristic usability evaluation is a method used to detect unclarities in the way an application is designed. Its aim is to identify usability issues before making recommendations to improve user experience. The main objective of us conducting such an evaluation was to understand, from a non-developers viewpoint, how well each part of our application comes together in an attempt to form an intuitive experience for an average user.

The application’s functionality is split into 5 main components: boards, lists, cards, tags and admin commands. Upon starting the application, users are prompted with entering the server port (Figure 1), after which they are presented with a window with all the active and joinable boards, also being able to create a new one (Figure 2). An admin sees an additional menu where they can edit an existing board’s properties - name, password protection - or delete one (Figure 3). Once a board is joined, the application shows lists of tasks (Figure 4). In this view, there are contrasting buttons that can add a list or return the user to any of the previously visited pages - server select or board overview. After clicking on any of the cards, the user is presented with a menu where they can edit a card’s properties: title, tags, description, subtasks, deadline, and colour (Figure 5). Editing the tags or colour of a card is done by clicking on a “+” sign next to the corresponding label. These actions each have a menu that will appear on the screen. Tags have fields for their name, whether they are active on the given card and their colour. New tags can be added within the same menu (Figure 6), and these tags will be available for use on all cards on the same board. Choosing a colour for both the tags and the cards is done with a colour picker (Figure 7) that presents a spectrum from which the user can choose the aspect of the item they are targeting.