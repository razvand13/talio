\section{Results}

Every expert succeeded in joining a board by id. It has to be taken into account that the form did not specify that the user should fill in the key of the server in order to access it. This, however, was not an impediment for the reviewers in fulfilling the task given. One professional, in particular,  explained an incorrect course of action by mentioning the first thing that came to mind when entering the board overview scene was to press the “+” button at the bottom of the window, instead of the Join button. This is a result of the high contrast in the visuals which postponed the completion of the task. When asked about the difficulty of entering a board by ID, 4 out of 5 experts carried the task without problems, whilst one of them considered it being a 2 out of 5 on a difficulty scale. Another comment mentioned that the boards in the middle should have an explanation of what they represent. Multiple experts suggested presenting the boards in descending order of interaction. When tasked with switching to another board, every expert carried it out successfully.

Through the exercises, one expert mentioned being inconvenienced by the design of the buttons, as the “Join” button in the board overview was smaller than the “Switch server” and “Admin view” buttons. Another professional suggested that there should be a separate window where users can choose if they want to create a board or join an existing one.

The tasks which involved creating and deleting a list presented satisfactory results, as each expert mentioned a correct way to delete a list by pressing the “Remove” button on a list or adding a list by clicking on the “+” button next to the lists respectively. In the exercise of creating a list, the respondents successfully noticed the button next to which it is written “New list” and pressed it intuitively. In both deleting and creating a task, 3 out of 5 experts mentioned the path they took, and the rest wrote just what button they pressed. Asked about the visual design of the list, one professional was in favour, three mentioned it is a 4 out of 5 on appearance and one considered it neutral.

The results for editing a board name as the administrator had spread results. Only three people figured out the correct path. One of the respondents mentioned being doubtful about having chosen the right path from the board itself. The other two experts were confused about the task and pressed the wrong button, the one next to the board name. When asked to review this exercise’s difficulty, one expert found it fairly difficult, two scaled it as neutral, one said that it was manageable and the last one wrote it was easy.

During the addition of a new card, one respondent mentioned that, although he pressed the add button, nothing changed. Additionally, one respondent was confused about not being able to find the “Add new card” button, since it took him directly to the  “Edit card” window. Whilst deleting card 3 from list 1, all the experts mentioned the correct way to delete the respective card. When asked about the design of the card overview page, one individual expressed it is neutral, two rated it a 4 out of 5 on appearance and two said it was nicely arranged.

In this section, we also received feedback for the colour customisation page. According to one of the respondents, the button for colour customization blends in with the delete and save buttons at the bottom of the card view, which was considered a poor design choice. The same evaluator had issues with having a larger and different button for the colour picker in the card view than in the tag view page. They suggested changing the visual design of the card customization pages in such a way that the design becomes consistent throughout these scenes. According to another response, the colour picker only has limited options. 

While being presented with the task of adding a new tag to a card with a chosen colour, 4 out of 5 respondents found executing this task easy, while one of them rated it neutral. One evaluator found issues with the flow of the application that makes the tag customizing feature possible. They expected that after picking a colour for the tag, they would be taken back to the tag customization page instead of the card editing scene. They suggested paying attention to setting the font colour of the tag according to the picked tag background colour, so the tag name remains readable regardless of colour choice. All evaluators rated the visual design of the tag customization page positively.

Each respondent found the resetting of the server doable since 3 out of 5 responses rated carrying out the task neutral on a scale from ‘very easy’ to ‘very hard’, whilst the two other respondents considered it easy. All evaluators took the expected steps to execute the task. Another task related to the admin side of the application was removing the password of a board that has password protection. In a similar manner, 3 out of 5 responses gave a neutral rating for the difficulty, while the other 2 found the process easy. In the current application, the user has to manually input the id of the board of which they want to edit the attributes, which was found cumbersome by two evaluators. One of them suggested having the option to edit the board directly in the admin board by clicking on it, instead of entering the details into a separate text field. Another response suggested having access to admin features in the lists view of a certain board, not just in the admin view, to omit to have to take the path between the board and the admin page each time. The visual design of the admin page also received criticism, an evaluator considered the scene too crowded and suggested the use of intermediate scenes.
