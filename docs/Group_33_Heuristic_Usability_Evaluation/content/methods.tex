\section{Methods}

In order to have an unbiased review of our own work, we recruited 5 experts that could help us find defective parts in our application’s design. Although their expertise in Java application was at beginner level, the feedback provided from an external eye on the project’s design, contributed towards upgrading the overall quality of the appearance.
	   
Our group provided the experts with a Figma application where we displayed how our application would ideally look and work, with not only the designs but also transitions depending on the user’s button selection. Alongside the prototype, we gave the experts a form with 3 types of questions. The first type consisted of a short task to complete, in order to get familiar with the application, for example, ‘Delete a List’. We formulated these tasks in such a way that the expert testing our application would not just be receiving direct instructions. Instead, they were asked to carry out common tasks that often involved multiple steps. The experts would then be asked to describe what they did and what path they took to complete these tasks. The second type of question was a rating out of 5 on either the difficulty of the task or the visual design of the concerned page. This helps us receive a concrete rating on each component of the design so that we know which parts were appreciated, and which had possible improvements. The last type of question is for them to provide feedback or suggestions on functionality and design elements. This was our most insightful type of question, as their answers explicitly showed us what our plan was lacking. 

In this exercise, the experts are using different types of heuristics. First of all, they used their error prevention skill by detecting mistakes that would lead to errors in the prototype. Most importantly, they practised their skill of giving critical and useful feedback on a completely unknown project that they haven’t worked on. On top of giving this feedback, they had to keep their suggestions consistent, as the application has to remain coherent in order to make the user’s experience as spontaneous as possible. 

The short paragraph answers where the expert describes the path they took to complete certain tasks provide us with important insight into how the user navigates our application. Our goal was to assess button placement and the overall layout of our application. The questions that asked for feedback on their experience helped us identify improvements we should make and measure the changes that still need to be made. The number scale question offered us valuable statistics on how well certain functionalities were implemented, and the visual appeal of our design. Because this type of question was included with every scene and most functionality, we obtained an extensive evaluation across almost all features. With these questions, we were able to measure the intuitivity, the potential ambiguity and the visual appeal of our application.
